\documentclass[12pt]{article}
\parindent0em
\parskip 1ex plus 0.4ex minus 0.4ex

\usepackage[a4paper,vmargin=30mm,hmargin=25mm]{geometry}
\usepackage{polyglossia}
\setdefaultlanguage{german}
\usepackage{fontspec}
\usepackage{lipsum}
\usepackage{xcolor}
\usepackage{listings}
\usepackage{amssymb}

\definecolor{lstbackground}{rgb}{0.95,0.95,1}      % hellgruener Rahmen
\lstset{language=Python}

\lstset{
  basicstyle=\small\ttfamily,
  backgroundcolor=\color{lstbackground},
  keywordstyle=\bfseries\ttfamily\color{blue},
  stringstyle=\color{orange!50!black}\ttfamily,
  commentstyle=\color{gray}\ttfamily,
  showstringspaces=false,
  flexiblecolumns=false,
  tabsize=4,
  numbers=left,
  numberstyle=\tiny,
  numberblanklines=true,
  stepnumber=1,
  numbersep=10pt,
  xleftmargin=15pt,
  literate=%
  {Ö}{{\"O}}1
  {Ä}{{\"A}}1
  {Ü}{{\"U}}1
  {ß}{{\ss}}1
  {ü}{{\"u}}1
  {ä}{{\"a}}1
  {ö}{{\"o}}1
  {~}{{\textasciitilde}}1
}

\begin{document}

\begin{center}
  \textbf{\LARGE Sichere Programmierung} \\[1ex]%
  \textbf{\Large Projekt 1}\\[2ex] %
  Julian Sobott \\ %
  (76511) \\ %
  David Sugar \\ %
  (76050) \\ %
  
\end{center}

% ****************************************************************************
\section{Zu Aufgabe 1}
% ****************************************************************************
Aus der Aufgabenstellung war gegeben, dass die Funktion \texttt{decode(text)}, die Buchstaben des übergebenen Textes in entsprechende Zahlen aus $\mathbb{Z}_{26}$ umwandeln und diese dann in einer Liste zurückgeben soll. Daraus ergibt sich der Definitionsbereich $D = \{a,...,z\}$ und Wertebereich $W = \{0,...,25\}$ mit $f: D \to W$ für die Symbole und $decode(): D^{*} \to W^{*}$ für Wörter beliebiger Länge.  

$f: D \to W$ wird durch \texttt{alph\_to\_num} realisiert, einem Python \texttt{dict}, dass von ascii Kleinbuchstaben aufsteigend auf die Zahlen von Null bis 25 abbildet und wiederum innerhalb von \texttt{decode()} in einer Schleife verwendet wird um jeden einzelnen Buchstaben des übergebenen Textes umzuwandeln. Werte außerhalb des Definitionsbereiches werden vom gegebenen Algorithmus ignoriert.
\begin{lstlisting}
alph_to_num = {k:v for v , k in enumerate(string.ascii_lowercase)}
\end{lstlisting}




% ----------------------------------------------------------------------------
\subsection{Zu Aufgabe 2}
% ----------------------------------------------------------------------------
Die Funktion \texttt{encode(text)} stellt die Umkehrfunktion von \texttt{decode()} dar, für alle $w \in \{a,..z\}^{*}$. Sie nimmt als Eingabe eine Liste von Zahlen $a \in \mathbb{Z}_{26}$ und gibt eine entsprechende Zeichenkette (String) zurück.

Das Abbilden von Zahlen auf die entsprechenden Buchstaben wird durch $num\_to\_alph: \{0,..,25\} \to \{a,..,z\}$ realisiert.
\begin{lstlisting}
num_to_alph = {v:k for v , k in enumerate(string.ascii_lowercase)}
\end{lstlisting}

Um den String schlussendlich zu bauen benötigt es dann nur einen Einzeiler.
\begin{lstlisting}
"".join([ num_to_alph[d] for d in int_list ])
\end{lstlisting}

Dadurch, dass \texttt{decode()} und \texttt{encode()} jeweils Funktion und Umkehrfunktion darstellen ergibt sich: w = encode(decode(w)).

\end{document}

%%% Local Variables: 
%%% TeX-PDF-mode: t
%%% TeX-master: t
%%% coding: utf-8
%%% mode: latex
%%% TeX-engine: xetex
%%% End: 
