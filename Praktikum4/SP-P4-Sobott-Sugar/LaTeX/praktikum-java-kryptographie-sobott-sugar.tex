\documentclass[12pt]{article}
\parindent0em
\parskip 1ex plus 0.4ex minus 0.4ex

\usepackage[a4paper,vmargin=30mm,hmargin=25mm]{geometry}
\usepackage{polyglossia}
\setdefaultlanguage{german}
\usepackage{fontspec}
\usepackage{lipsum}
\usepackage{xcolor}
\usepackage{listings}
\usepackage{hyperref}
\usepackage{graphicx}
\definecolor{lstbackground}{rgb}{0.95,0.95,1}      % hellgruener Rahmen
\lstset{language=Python}

\lstset{
  basicstyle=\small\ttfamily,
  backgroundcolor=\color{lstbackground},
  keywordstyle=\bfseries\ttfamily\color{blue},
  stringstyle=\color{orange!50!black}\ttfamily,
  commentstyle=\color{gray}\ttfamily,
  showstringspaces=false,
  flexiblecolumns=false,
  tabsize=4,
  numbers=left,
  numberstyle=\tiny,
  numberblanklines=true,
  stepnumber=1,
  numbersep=10pt,
  xleftmargin=15pt,
  literate=%
  {Ö}{{\"O}}1
  {Ä}{{\"A}}1
  {Ü}{{\"U}}1
  {ß}{{\ss}}1
  {ü}{{\"u}}1
  {ä}{{\"a}}1
  {ö}{{\"o}}1
  {~}{{\textasciitilde}}1
}

\begin{document}

\begin{center}
  \textbf{\LARGE Sichere Programmierung} \\[1ex]%
  \textbf{\Large Projekt 4}\\[2ex] %
  David Pierre Sugar\\ %
  (76050) \\ %
  Julian Sobott\\ %
  (76511) \\ %
  
\end{center}

\tableofcontents

% ****************************************************************************
\section{Einleitung}
% ****************************************************************************
In diesem Praktikum geht es darum Konzepte, die in der Vorlesung \textit{Einführung in die IT-Sicherheit} erklärt wurden, in Java zu implementieren. Hierzu wird die Kryptographie-API von Java verwendet.

% ****************************************************************************
\section{Aufgabe 1: Kryptographische Hashfunktion}
% ****************************************************************************

\subsection{Hashing}

Eine Hashfunktion wandelt eine beliebige Eingabe in eine Ausgabe fester Länge um. Das Ergebnis wird als \textit{Hash} oder auch  engl. \textit{digest} bezeichnet.
Jede Hashfunktion, sollte dabei aber folgende Eigenschaften haben:
\begin{itemize}
\item \textbf{Einwegfunktion:} Das heißt der original Wert soll nicht aus dem Hash zurück gerechnet werden können.
\item \textbf{Kollisionsresistenz:} Da der Wertebereich der Funktion kleiner als der Definitionsbereich, kann es zu Kollisionen kommen. Das heißt unterschiedliche Eingabe Werte erzeugen den gleichen Ausgabewert. Eine Hashfunktion sollte möglichst solche Kollisionen vermeiden.
\item \textbf{Große Streuung:} Hashes von ähnlichen Werten sollen weit auseinander liegen. 
\item \textbf{Effiziente Berechnung:} Das Hashen einer Eingabe sollte möglichst effizient sein. Aufgrund der Einwegfunktion, ist das berechnen der Eingabe aufgrund eines Hashes, bei modernen Hashfunktionen aber nicht effizient lösbar.
\end{itemize}
Hashes können unter anderem dafür verwendet werden, um Prüfsummen (engl. checksums) zu berechnen. 
Diese Können dann dafür verwendet um die Integrität einer Nachricht zu bewahren. Das heißt es kann kontrolliert werden, dass die Nachricht nicht verändert wurde.

\subsection{Implementierung}
In dieser Aufgabe soll die Erzeugung und Verifikation einer solchen Prüfsumme implementiert werden. Als Hashalgorithmus soll hier \texttt{SHA-256} verwendet werden.
Für die Aufgabe wurden zwei statische Methoden implementiert. \texttt{byte[] createChecksum(String message)} und \texttt{boolean verifyChecksum(byte[] checksum, String message)}. Die erste erzeugt eine Prüfsumme und die zweite verifiziert ob ein String dieselbe Prüfsumme erzeugt.

Fürs Erzeugen wird eine \texttt{MessageDigest} Instanz, welche mit \texttt{SHA-256} arbeitet, geholt. Mit dieser wird dann, der String in einen entsprechenden Hash Wert umgewandelt.

\begin{lstlisting}
MessageDigest md = MessageDigest.getInstance("SHA-256");
return md.digest(message.getBytes());
\end{lstlisting}

Fürs Verifizieren, wird auf dieselbe Weise von dem neuen String die Checksum erstellt. Ob die beiden gleich sind, wird mit der statischen Methode \texttt{MessageDigest.isEqual()} überprüft.

\begin{lstlisting}
byte[] actual = createChecksum(message);
return MessageDigest.isEqual(actual, checksum);
\end{lstlisting}


\subsection{Beispiel}
Zur demosntration der beiden Methoden, wurden in der \texttt{main} Methode 2 Strings erstellt. Die Verifikation gibt true zurück, wenn beide Strings gleich sind und false, wenn sie unterschiedlich sind. Führt man das Programm erhält man die folgende Ausgabe

\begin{lstlisting}
Checksums are equal: true
Checksums are equal: false
\end{lstlisting}

Das ganze könnte mithilfe von Verschlüsselung zum Beispiel verwendet werden, um Nachrichten zu versenden deren Integrität gewahrt werden muss.


% ****************************************************************************
\section{Aufgabe 2: Symmetrische Verschlüsselung}
% ****************************************************************************
Symmetrische Verschlüsselung ist ein Weg, Nachrichten vor unerwünschten
Einblicken zu schützen. Symmetrisch heißt dabei, dass der selbe Key zum ver-/
und wieder endschlüsseln verwendet wird. Ein Beispiel hierfür wäre der \texttt{AES}
(Advanced Encryption Standard) Algorithmus, der auch \texttt{Rijndael}
Algorithmus genannt wird, in Anlehnung an die Erfinder Vincent Rijmen und Joan
Daemen. Symmetrische Verschlüsselungen bieten, einen angemessen großen Key
vorrausgesetzt, guten Schutz und eine schnelle Ver-/ Entschlüsselung, 
jedoch muss zur sicheren Übermittlung auf andere
Verschlüsselungsverfahren wie RSA zurückgegriffen werden, da diese sonst leicht
von Dritten abgegriffen werden können.

\subsection{Cipher}
Der AES ist ein Block Cipher, d.h. es werden jeweils Blöcke zu je \texttt{128 Bit}
verschlüsselt, dabei werden die \texttt{Key-Längen 128, 192 und 256 Bit}
unterstützt. Die Länge des Keys gibt dabei die Anzahl an Transformationsrunden
vor, die für die Verschlüsselung durchlaufen werden. Um den Text wieder zu
entschlüsseln, werden die Runden rückwärts durchlaufen. Die Transformationen
finden dabei auf einer 4x4 Byte Matrix statt.

\subsection{Implementierung}
Das Beispiel besteht darin, einen String zu verschlüsseln, den Cipher Text auf
Standard Out auszugeben, den Cipher Text wieder zu entschlüsseln und auch den
entschlüsselten Text auf der Kommandozeile auszugeben. Sollte schlussendlich der
anfangs verschlüsselte Text richtig ausgegeben werden, so kann davon ausgeganen werden,
dass die symmetrische Verschlüsselung mittels AES richtig ausgeführt wurde.

\subsubsection{Key erstellen}
Als erstes wird ein zufälliger Key erstellt, der dann zusammen mit dem AES
Cipher verwendet werden soll. Dafür wird die Methode \texttt{getAESKey()}
aufgerufen, die einen \texttt{AES Key Generator} instanziert und danach mittels
\texttt{init()} diesen Key Generator mit einer Key Größe von 256 initialisiert.
Danach wird mit \texttt{generateKey()} der zufällige Key erstellt und zurück
gegeben.

\begin{lstlisting}
SecretKey key = getAESKey();
\end{lstlisting}

\subsubsection{Initialisierungs-Vektor}
Ein Initialisierungsvektor ist eine Zufallszahl, die als weitere Sicherheit in
die Verschlüsselung mit eingebracht wird. Um das Brechen des Ciphers,
z.B. mittels Dictionary Attack, zu
erschweren wird der vorrausgegangene Cipher Block mit genutzt, um den derzeitigen
Plaintext Block zu verschlüsseln. Da für den ersten Plaintext Block noch kein
Cipher Block zur verfügung steht, der verwendet werden kann wird eine
Zufallszahl, der IV, zur Verschlüsselung hinzugezogen.

Dieser IV wird von \texttt{getIvSpec()} zurückgegeben. 

\begin{lstlisting}
IvParameterSpec ivSpec = getIvSpec();
\end{lstlisting}

Dabei nutzt die Methode
einen Pseudo-Zufallszahl-Generator um ein Byte Array zu befüllen. Dieses Array
wird dann als Argument genutzt, um mit \texttt{IvParameterSpec()} ein IV Objekt
zu erzeugen.

\subsubsection{Cipher}
Um den Cipher zu initialisieren, wurde die Methode \texttt{initCipher()}
implementiert. Dies erhällt \texttt{Key} und \texttt{IV}, sowie den Modus
(ENCRYPT, DECRYPT) als Parameter und instanziert ein AES Cipher Objekt, das dann zur
verschlüsselung verwendet werden kann.

\begin{lstlisting}
Cipher cipher = initCipher(key, Cipher.ENCRYPT_MODE, ivSpec);
\end{lstlisting}

\subsection{Verschlüsselung}
Mittels \texttt{encrypt()} kann nun der Plain Text verschlüsselt werden. Dazu
werden Cipher sowie Text and die Methode übergeben, zurück kommt der Cipher Text
als Byte Array. Intern wird einfach nur \texttt{doFinal()} auf dem Cipher Objekt
aufgerufen.

\begin{lstlisting}
byte cipherText[] = encrypt(cipher, pt);
\end{lstlisting}

\subsubsection{Ausgabe Cipher Text}
Die Methode \texttt{printCipherText()} erhällte das Byte Array als Argument und
gibt es in \texttt{Base 64} codiert auf der Kommandozeile aus.

\begin{lstlisting}
printCipherText(cipherText);
\end{lstlisting}

\subsubsection{Entschlüsselung}
Nun muss der Text wieder entschlüsselt werden. Dazu wird das Cipher Objekt nun
mit dem \texttt{DECRYPT} Mode initialisiert, dabei bleiben Key und IV die
selben. Danach wird \texttt{decrypt()} aufgerufen. Die Methode ruft erneut
\texttt{doFinal()} auf und entschlüsselt damit den übergebenen Cipher Text,
zurück in den Ausgangstext.

\begin{lstlisting}
cipher.init(Cipher.DECRYPT_MODE, key, ivSpec);
String new_pt = decrypt(cipher, cipherText);
\end{lstlisting}

\subsubsection{Ausgabe Plain Text}
Schlussendlich wird nun auch der entschlüsselte Plain Text auf der Kommandozeile
ausgegeben.

\begin{lstlisting}
System.out.println("Plain Text: " + new_pt); 
\end{lstlisting}

\subsection{Beispielausgabe}

\begin{lstlisting}
[sugar@jellyfish Code]$ java SymEncrypt 
Cipher Text: lPSGD5vFeGS7KlTBZjX2RQ==
Plain Text: Neo... Wake up!
[sugar@jellyfish Code]$ java SymEncrypt 
Cipher Text: OG5fVtza0u6P5RybQLLMGA==
Plain Text: Neo... Wake up!
[sugar@jellyfish Code]$ java SymEncrypt 
Cipher Text: +wCZzQ9bfxIA9r1aGFjB8Q==
Plain Text: Neo... Wake up!
\end{lstlisting}


% ****************************************************************************
\section{Aufgabe 3 Asymmetrische Verschlüsselung}
% ****************************************************************************

\subsection{Asymmetrische Verschlüsselung}

Eine Asymmetrische Verschlüsselung wird dann verwendet, wenn der Schlüsselaustausch nicht geheim erfolgen kann. Es gibt ein Schlüsselpaar, welches aus einem \textbf{private key} und einem \textbf{public key} besteht. 
Der public key kann für jeden zugänglich sein. Der private key muss aber privat bleiben und darf nicht von einer zweiten Instanz gekannt werden. Für den Nachrichtenaustausch, erstellt zuerst eine Instanz, ein solches Schlüsselpaar. Der public key wird dann an den Kommunikationspartner weiter geschickt. Dies muss nicht verschlüsselt sein, da der public key kein Geheimnis enthält. Der Kommunikationspartner kann nun eine Nachricht mit dem public key verschlüsseln und zurück senden. Diese Nachricht kann nur mit dem passenden  private key entschlüsselt werden.
Dieses verfahren kann zum beispiel verwendet werden, um symmetrische Schlüssel auszutauschen. Asymmetrische Verschlüsselung ist im Vergleich zur symmetrischen langsamer und wird deshalb meist nur zu Beginn einer Kommunikation verwendet.

\subsection{Implementierung}

Diese Aufgabe soll die Nutzung des \texttt{RSA} Algorithmus für die asymmetrische Verschlüsselung zeigen. Ein Text soll mit einem public key verschlüsselt werden, und mit dem entsprechendem private key wieder entschlüssselt werden.
Hierzu wird als erstes mithilfe eines \texttt{KeyPairGenerator} ein Schlüsselpaar für die Verschlüsselung mit RSA erstellt. Auf die beiden Schlüssel kann dann mithilfe von gettern zugegriffen werden.

\begin{lstlisting}
// Schlüsselgenerierung
KeyPairGenerator keyPairGen = KeyPairGenerator.getInstance("RSA");
keyPairGen.initialize(keysize);
keyPair rsaKeyPair = keyPairGen.generateKeyPair();

// private und public key
PublicKey publicKey = rsaKeyPair.getPublic();
PrivateKey privateKey = rsaKeyPair.getPrivate();
\end{lstlisting}


Als nächstes wird ein Objekt, welches für die Ver- und Entschlüsselung verantwortlich ist, mithilfe der factory Methode \texttt{getInstance} geholt. Dieses kann mehrmal verwendet werden, muss aber vor jeder Verwendung initialisiert werden. Hierbei muss übergeben werden, ob ver- oder entschlüsselt werden soll, und der jeweilige passende key.

\begin{lstlisting}
Cipher rsa = Cipher.getInstance("RSA");

// Verschlüsseln
rsa.init(Cipher.ENCRYPT_MODE, publicKey);

// Entschlüsseln
rsa.init(Cipher.DECRYPT_MODE, privateKey);
\end{lstlisting}

Mit der \texttt{doFinal} Methode, kann dann der ver-/entschlüsselungs Prozess abgeschlossen werden. Hierbei wird immer ein \texttt{byte[]} zurückgegebn, welches den Text enthält.

\begin{lstlisting}
// Verschlüsselter Text
byte[] cipherText = rsa.doFinal(plainText.getBytes());

// Entschlüsselter Text
byte[] plainTextAgain = rsa.doFinal(cipherText);
\end{lstlisting}

Um den verschlüsselten Text schöner auf der Konsole anzuzeigen, wird er noch in Base64 umgewandelt. Außerdem wird das \texttt{byte[]} an den Konstruktor der Klasse \texttt{String} übergeben, um den Text als zusammenhängenden Text und nicht als Array auszugeben.

\begin{lstlisting}
Base64.Encoder encoder = Base64.getEncoder();
String message = encoder.encodeToString(cipherText);
System.out.println("encrypted message: " + message);
\end{lstlisting}

\begin{lstlisting}
System.out.println("decrypted message : " + new String(plainTextAgain));
\end{lstlisting}

Eine Ausgabe kann dann Beispielsweise wie folgt aussehen:
\begin{lstlisting}
\begin{lstlisting}
System.out.println("decrypted message : " + new String(plainTextAgain));
\end{lstlisting}

\begin{lstlisting}
Original message: Kryptographie macht immer noch Spass!!!

encrypted message: GyHYxvw3nZiXt7LbqJAYPgn7idQKFRiawHbA7
D6Qgq+eDOdHznt7dBc9+bTkqlCUvxvHUTZt4XPa+hYyKCfrdH/UI/NSm
9rxjRIjdy3OcpBdA+6RqLga5jNIEPd9AJvhfLMUVbsHHY
+UQVoV4Fa5FX5TIEAfo4VQUXIvjmHngl5t6nNqdmjBqt4i8
gsrO0ai2HYYJdowbkzj6lwu7sZPByCGibahXaO11GQ/eQUX83Z2K
Kacazn/7KeG8HLhsNAUVp4TqJmRBEpOZwN2/CC0dDvMhXjZdvhI7zYd
aOkXOf+hw0xbHZ2ahNdjPfMmYCLS4OWfSK0sDiaLbA==

decrypted message : Kryptographie macht immer noch Spass!!!
\end{lstlisting}

% ****************************************************************************
\section{Aufgabe 4: Digitale Signatur}
% ****************************************************************************
Digitale Signaturen werden genutzt, um die Authentizität, Integrität,
Überprüfbarkeit und Nicht-Abstreitbarkeit von digitalen
Dokumenten sicher zu stellen, gewährleisten aber in keinem Fall die Vertraulichkeit eines
Dokuments. Zur digitalen Signatur eignen sich dabei die meisten
asymmetrischen Verschlüsselungssysteme, wie z.B. RSA. RSA ist dabei aber anfällig für
Attacken, sollte der selbe Text mehrmals mit dem Selben Schlüssel signiert
werden.

\subsection{Verfahren}
Anfangs wird eine zufälliges Schlüsselpaar, bestehend aus Public-Key und
Private-Key, erstellt. Das gewünschte Dokument wird dann zuerst gehased 
und der Hash danach mittels RSA unter hinzunahme des Private-Key verschlüsselt.
Danach kann der Hash an das Dukument mit angehängt werden. Um die Authentizität
des Dokuments nun zu verifizieren kann jeder, der den Public-Key besitzt, den
Hash entschlüsseln, selber nocheinmal das Dokument hashen und dann beide Hashes
miteinander vergleichen. Stimmen diese überein, so kann davon ausgegangen
werden, einen kollisionssicheren Hashing-Algorithmus vorrausgesetzt, dass das
Dokument nicht verändert wurde und dass der Author tatsächlich die Person, mit
dem zugehörigen Private-Key ist.

\subsection{Implementierung}
Das Beispiel besteht darin, einen Text mittels RSA digital zu signieren. Die Signatur
wird dann auf der Konsole ausgegeben. Danach wird sie verifiziert, um zu
überprüfen ob das Dokument authentisch ist und tatsächlich vom Absender stammt. 
Das Ergebnis wird ebenfalls wieder auf der Kommandozeile ausgegeben. 
Erwarte wird, dass die Prüfsummen beide gleich sind, d.h. dass verify()
\texttt{true} zurückliefert.

\subsection{Key-Pair generieren}
Der Code zur generierung des Schlüsselpaares wurde in der Methode
\texttt{genKeyPair()} gekapselt. Diese generiert mittels
\texttt{KeyPairGenerator} einen RSA Private- und Bublic-Key mit Schlüsselgröße 20000
und gibt das Paar als Rückgabewert zurück.

\begin{lstlisting}
KeyPair kp = genKeyPair();
\end{lstlisting}

\subsection{Text signieren}
Das Signieren eines beliebigen Textes geschieht mittels \texttt{sign()}. Die
Methode erhält den Text als String, sowie den privaten Schlüssel als Argumente
und liefert eine Signatur des Textes zurück.

\begin{lstlisting}
byte[] signature = sign(pt, kp.getPrivate());
\end{lstlisting}

Die Methode selber instanziert als erstes eine SHA256, RSA Instanz von
\texttt{Signature} (1). Diese wird dann mit dem privaten Schlüssel initialisiert
(2). Danach wird mittels \texttt{update()} der zu signierende Text, als Byte
Array übergeben (3).
Schlussendlich wird der Text dann signiert (4).

\begin{lstlisting}
Signature rsa = Signature.getInstance("SHA256withRSA");
rsa.initSign(pk); // initialize this object fo signing
rsa.update(pt.getBytes()); // update the data to be signed
byte signature[] = rsa.sign();  // sign data
return signature;
\end{lstlisting}

\subsection{Signatur verifizieren}
Um die Authentizität und Integrität des Dukuments zu überprüfen, kann nun
die \texttt{verify()} methode zusammen mit der Signatur und dem Public-Key
aufgerufen werden.

\begin{lstlisting}
verify(pt, signature, kp.getPublic())
\end{lstlisting}

Als erstes wird wieder eine SHA256, RSA Instanz von Signature erzeugt (1).
Diesmal wird die Instanz jedoch mit dem Public-Key initialisiert (2).
Danach wird der Text mittels \texttt{update()} an die Instanz zur Verifikation
übergeben (3). Schlussendlich wird das Dokument unter Verwendung der Signatur
verifiziert (4). Sollte Verifikation erfolgreich sein wird \texttt{true} zurück
gegeben, andernfalls \texttt{false}.

\begin{lstlisting}
Signature sigInst = Signature.getInstance("SHA256withRSA");
sigInst.initVerify(pk); // initialize this object for verification
sigInst.update(pt.getBytes()); // updates the data to be verified
return sigInst.verify(signature);
\end{lstlisting}

\subsection{Beispielausgabe}

\begin{lstlisting}
$ java SignatureTest 
Text: Follow the white rabbit
Signature: wtTJTF8X9UMrSRsWl44M8t6ui1aHTw+7EGOLgM
+fxjWRdVyLofLyq25VWNGSTlQ98A9Pdg4ywRvfaOjtPgUEC1T
OyUfk4hfzK+S8TmplYwVAlIGdInDvGB1u/lV751RKe/8JcCwv
kYHwsjBh0JB06OkDyRLip07STl2mPCYTS4UqJ+h7Dk7Cqh239
kr8WyrA7Z9Ni+vy4Y8YPfvgxmWEYwwYaZG77O2DqXnz1TfogN
8ztEMMEx9AoXGESw3k6GoB+Q3DwTpngVOLrBOTaGXWTEzys3k
GrHJEZkhNRfRGirqVzOG9RJfLpLeu6l1jQxjCWwDUekj8qbiB
M6lLGKoBcg==
Verifying document...
Verified: true


Verändere Text...
Text: Follow the white rabbit clown
Verifying document...
Verified: false
\end{lstlisting}
Das Beispiel führt das oben beschriebene Verfahren durch und signiert den
gegebenen Text. Danach wird der Text mittels Signatur verifiziert, siehe
\texttt{Verified: true}. Nun wird der Text leicht verändert und danach wird
versucht ihn ein zweites mal zu verifizieren. Diesmal stimmen die Hashes jedoch
nicht überein, wodurch der Text nicht verifiziert werden konnte.

\end{document}



%%% Local Variables: 
%%% TeX-PDF-mode: t
%%% TeX-master: t
%%% coding: utf-8
%%% mode: latex
%%% TeX-engine: xetex
%%% End: 
